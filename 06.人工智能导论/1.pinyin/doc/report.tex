\documentclass[12pt, UTF8, a4paper]{ctexart}
\usepackage{morelull}
\usepackage{graphicx}
\usepackage{booktabs}
\usepackage{threeparttable}

\title{拼音输入法\ 实验报告}
\author{2018011365 张鹤潇 731931282@qq.com}
\date{\today}

\begin{document}

\begin{titlepage}
    
\maketitle

%\begin{abstract}
%    
%\end{abstract}

\tableofcontents

\thispagestyle{empty}
\end{titlepage}

%\newpage
\setcounter{page}{1}
\section{基本算法}

\subsection{HMM}

拼音输入法的核心在于建立汉语语言模型,即对汉字序列出现的概率进行建模。具体地说,
记要预测的汉字序列$W=<w_1,\cdots,w_n>$,注意,这里的$w$不仅包括汉字,
还包括起始符和终止符。 我们要估计:
\begin{equation}\label{eq:1}
\mathop{argmax}\limits_{w\in W} P(w_1,\cdots,w_n)=
\mathop{argmax}\limits_{w\in W} P(w_1)P(w_2|w_1)\cdots 
P(w_n|w_1,\cdots, w_{n-1})
\end{equation}

为了简化上式,假设该问题满足$k$阶马尔可夫性质,
即每个字出现的概率仅与之前$k$个字有关,则$k=1$时,问题简化为:
\begin{equation}\label{eq:2}
\mathop{argmax}\limits_{w\in W} P(w_1,\cdots,w_n)=
\mathop{argmax}\limits_{w\in W} P(w_1)
\prod_{i=2}^n P(w_i|w_{i-1})
\end{equation}

用频率估计概率,则:
$$
P(w_i|w_{i-1})=\frac{\#(w_{i-1}w_i)}{\#w_{i-1}}
$$

$\#(w_{i-1}w_{i})$表示$w_{i-1}w_i$在训练语料中出现的次数。

这就是bigram(二元语法)模型。$k=2$时,称为trigram(三元语法)模型。

\subsection{Viterbi算法}

用动态规划的方法最优化\ref{eq:2}式。记序列$W$中第$i$个字的第$j$个候选为$w_{ij}$,
长度为$i$,结尾为$k$的最佳前缀为$T_{i,j}$,则:
\begin{equation}\label{eq:3}
P(T_{i,j}) = \begin{cases}
\max\limits_{k} \{P(T_{i-1,k})P(w_{i,j}|T_{i-1,k})\} &i\ge 2\\
P(w_{i,j})& i=1
\end{cases}
\end{equation}
$$
\mathop{argmax}\limits_{w\in W} P(w_1,\cdots,w_n)
=\mathop{argmax}\limits_{k} P(T_{n,k})
$$

设平均情况下,每个拼音对应$c$个汉字,则上述算法时间复杂度为$O(cm)$,对于$c$和$m$都是线性的,
效率极高。然而,它只能在二元模型下高效地计算出最优解,在三元模型下的优化见下文。

\section{成果展示}

本代码支持训练任意n元字模型。由于开发环境内存所限,
在本机上只训练到三元字模型为止。使用和复现方法见\href{../readme.md}{readme}.

在2016年11月的新闻语料中随机抽取了1000句话,近万字作为验证集。

\begin{table*}[htbp]
\centering
\begin{threeparttable}
    \caption{\label{tab:1}新闻语料测试结果} 
    \small
    \begin{tabular}{cccccc} 
     \toprule 
     模型 & $\gamma$ & band width & 字准确率& 句准确率 & 运行时间(s) \\ 
     \midrule 
    3-gram\tnote{1}  & 200   & 6    & 95.22 \%  & 73.20 \%  & 185.9\\
    3-gram with low bias\tnote{2} & 1   & 6  & 97.76 \%  & 85.70 \%  & 195.0\\
    3-gram with low bias & 200 & 6 & 97.24 \% & 83.80 \% & 195.5\\
     \bottomrule 
\end{tabular}
\begin{tablenotes}
    \footnotesize
    \item[1] 用于调参的模型,训练语料为11月前的新闻。
    \item[2] 全模型,训练集除新闻外,包括了来自微信公众号的\href{https://github.com/nonamestreet/weixin\_public\_corpus}{训练语料}。
\end{tablenotes}
\end{threeparttable}
\end{table*}

以全模型在同学们贡献的测试集(500多句, 逾5000字)上测试, 结果与验证集相差无几,
可见模型泛化性能的优秀。

\begin{table}[htbp] 
    \caption{\label{tab:2}多样化语料测试结果} 
    \centering
    \small
    \begin{tabular}{cccccc} 
     \toprule 
     模型 & $\gamma$ & band width & 字准确率& 句准确率 & 运行时间(s) \\ 
     \midrule 
     3-gram  & 200   & 6    & 94.21 \%  & 74.07 \%  & 92.5 \\
     \bottomrule 
    \end{tabular}
\end{table}



\section{优化细节}

\subsection{训练语料预处理}

原始语料中含有大量字母、数字、标点等非汉字字符,先对它们进行预处理。

\begin{itemize}
    \item 根据断句标点(如逗号,句号,冒号等)将文章分成多句。
    \item 将每句话中的阿拉伯数字转换成中文数字,去掉剩下的所有非汉字字符。
    \item 在句首句尾添加\#\#标记。
\end{itemize}

\subsection{句首尾概率的修正}

用"\#\#"对句首句尾作标记,对句首句尾字出现的概率做修正。

\subsection{多音字的区分}

简单测试之后,我发现许多错误是由字音误判导致的。比如,
算法会将"yin xing"转换成银行。为了解决这个问题,
我在预处理中用pypinyin对原始语料进行了注音。将一个字的不同读音用数字区分开,
如行(xing)标记为"行0",行(hang)标记为"行1"。
当然训练和预测的算法也要在细节上做相应的修改。

至此,预处理前后语料文本的变化举例如下:

\begin{lstlisting}[language=js]
 // 预处理前
 4月26日第二代乐视超级手机京东首售
 // 预处理后
 ##四0月0二0十0六0日0第0二0代0乐0视0超0级0手0机0京0东0首0售0##
\end{lstlisting}

当然,pypinyin注音并不完美,比如它会将“添砖加瓦”标注为“tian zhuan jie wa”.
我觉得手工修复这些错误太繁琐了,并不值得。很遗憾,对于这个问题,我还没有找到解决办法。

\subsection{对线性平滑算法的改进}

在三元模型中,由于语料的规模限制,有一些三元组在语料中可能没有出现过,
或者出现的次数很少,如果直接取$P(w_{i}|w_{i-2}w_{i-1})$为概率估计值,
会导致整句概率过小甚至为0。一个简单的解决的办法是线性地组合模型:
$$
P(w_i|w_{i-2}w_{i-1})_{smooth}=
\lambda_1 P(w_i|w_{i-2}w_{i-1})+\lambda_2 P(w_i|w_{i-1}) 
+(1-\lambda_1-\lambda_2)P(w_i)
$$

但是这就引入了对两个超参数的调节,对其简单改进如下:

\begin{eq}
    \lambda_1&= \frac{\#(w_{i-2}w_{i-1})}{\#(w_{i-2}w_{i-1})+\gamma}\\
    \lambda_2&= (1-\lambda_1)\frac{\#w_{i-1}}{\#w_{i-1}+\gamma}\\
\end{eq}

这样就将需要调节的超参数减少到了一个。对该方法的直观解释是,
$w_{i-1}w_i$出现的次数越多,$\lambda_1$就会越趋近于1,
三元模型就会占主导地位,而二元、一元模型的权重就会减少。


\subsection{对Viterbi算法的改进}

上文提到,\ref{eq:3}式仅在二元模型下能找到最优解,
在三元模型中,Viterbi算法应该变为:
\begin{equation}\label{eq:4}
P(T_{i,j}) = \begin{cases}
\max\limits_{k,h} \{P(T_{i-2,k})P(w_{i-1,h}|T_{i-2,k})P(w_{i,j}|T_{i-1,h})\} &i\ge 3\\
\max\limits_{k} P(w_{i,j}|T_{i-1,k})& i=2\\
P(w_{i,j})&i=1
\end{cases}
\end{equation}

现在算法的时间复杂度数量级变为$O(c^2m)$,
实测表明程序的运行效率降低了十几倍,这几乎是不可接受的。

一个更好的选择是选用近似算法。
对\ref{eq:3}式略作改动,对于第$i$个候选字,
维护前$B$(band width)个概率最大的序列组合,而不只是最大者。算法时间复杂度$O(Bcm)$.
\begin{equation}\label{eq:5}
P(T_{i,j}) = \begin{cases}
\mathop{topB}\limits_{k} \{P(T_{i-1,k})P(w_{i,j}|T_{i-1,k})\} &i\ge 2\\
P(w_{i,j})&i=1
\end{cases} 
\end{equation}

现在$P(T_{i,j})$表示由$B$个候选序列组成的集合,
这就大大减少了因为\ref{eq:3}的贪心剪枝而得不到最优解的情况。

注意到,当$B=1$时,\ref{eq:5}式退化为\ref{eq:3}式;
$B=c$时,\ref{eq:5}式近似为\ref{eq:4}式。
该近似算法在性能和效率上做了平衡,是非常重要的优化。

最后还有一点:取对数加速概率计算,防止数值下溢。
\begin{eq}
\mathop{argmax}\limits_{w\in W} P(w_1,\cdots,w_n)&=
\mathop{argmax}\limits_{w\in W} P(w_1)\prod_{i=2}^n P(w_i|w_{i-1})\\
&=\mathop{argmax}\limits_{w\in W} \log{P(w_1)}+\sum_{i=2}^n \log{P(w_i|w_{i-1})}
\end{eq}

\subsection{测试效果}

综上所述,在交叉验证集上测试,
各种优化的效果对比如下:

\begin{table}[htbp] 
    \caption{\label{tab:3}模型优化测试结果($\gamma=10$)} 
    \centering
    \footnotesize
    \begin{tabular}{cccccccccc} 
     \toprule 
     ID & 模型 & 首尾修正 & 多音字 & 
     band width & 字准确率& 句准确率 & 运行时间(s) \\ 
     \midrule 
     1    & 2-gram & 有       & 有     & 1    
     & 90.21\%   & 52.10\%   & 25.5\\
     2    & 3-gram & 无       & 无     & 1    
     & 92.02\%   & 62.20\%   & 32.5\\
     3    & 3-gram & 有       & 无     & 1    
     & 92.93\%   & 66.20\%   & 32.1\\
     4    & 3-gram & 有       & 有     & 1    
     & 93.57\%   & 67.90\%   & 34.2\\
     5    & 3-gram & 有       & 有     & 5    
     & 94.79\%   & 71.50\%   & 158.6\\
     \bottomrule 
    \end{tabular}
\end{table}

\begin{table}[htbp] 
    \caption{\label{tab:4}模型预测效果对比}
    \footnotesize
    \centering
    \begin{tabular}{p{60pt}p{60pt}p{60pt}p{60pt}p{60pt}p{60pt}}
    \toprule
    Answer & <1> & <2> & <3> & <4> & <5>\\
    \midrule
    一旦复发常造成终身残疾乃至死亡 & 一旦复发\textbf{场}造成\textbf{中省}残疾乃至死亡 & 一旦复发\textbf{肠}造成终生残疾乃至死亡 
    & 一旦复发\textbf{肠}造成终生残疾乃至死亡 & 一旦复发\textbf{肠}造成终生残疾乃至死亡 & 一旦复发\textbf{畅}造成终生残疾乃至死亡\\
    这大概是我听过的最舒心的音乐会 & \textbf{浙}大概是我听过的最舒心的音乐会     & \textbf{着}大概是我听过的最舒心的音乐会 
    & \textbf{浙}大概是我听过的最舒心的音乐会 & \textbf{浙}大概是我听过的最舒心的音乐会 & 这大概是我听过的最舒心的音乐会\\
    夏山苍翠而如滴                 & \textbf{下山}苍翠而入低                     & \textbf{下山}苍翠而如滴
    & \textbf{下}山苍翠而\textbf{入的}             & \textbf{下山}苍翠而如滴                 & 夏山苍翠而如滴\\
    到后来祭祀人格化的神灵         & 到后来祭祀人格化的\textbf{申领}             & 到后来\textbf{其四}人格化的神灵
    & 到后来\textbf{其四}人格化的神灵         & 到后来\textbf{急死}人格化的神灵         & 到后来\textbf{寄私}人格化的神灵\\
    \bottomrule
    \end{tabular}
\end{table}

二元模型<1>只能考虑相邻一个词的关系,而汉语中频繁出现的二字词又太多,这就容易出现奇怪的结果;
不考虑多音字的模型<2, 3>容易混淆常见字的读音;剪枝过早的模型<4>在长句中的表现受限。
但即便是经过了大量优化的<5>,在相邻字之间没有明显承接关系时也表现不佳,这是n-gram模型本身的问题。


\section{参数调节}

\subsection{gamma的选取}

在区分多音字的三元模型(band width = 1)下测试。

\begin{table}[htbp] 
    \caption{\label{tab:5}gamma的选取} 
    \centering
    \footnotesize
    \begin{tabular}{cccccccccc} 
     \toprule 
     gamma    & 0.1     & 1       & 10      & 100     & 
     150     & 200     & 250    & 300     & 500\\
    \midrule
     字准确率 & 93.15\% & 93.34\% & 93.57\% & 93.99\% & 
     94.12\% & 94.19\% & 94.11\% & 94.15\% & 94.04\%\\
     句准确率 & 66.50\% & 67.00\% & 67.90\% & 68.50\% & 
     68.20 \% & 68.20\% & 68.20\% & 68.40\% & 67.80\%\\
     \bottomrule 
    \end{tabular}
\end{table}

测试中发现,在一定范围内,增大$\gamma$能提高模型的泛化能力,
即提高模型在训练集中没有的测试语料上的预测性能。见\ref{tab:1}。

这个现象很容易理解:$\gamma$越大,模型对三元字的依赖就越小,而汉语中的三元组实在太多了,
其中大部分在预测语料中没有出现过。
可见,在这个场合下,平滑也起到了防止过拟合的作用。

\subsection{band width的选取}

在区分多音字的三元模型($\gamma = 200$)下测试。

\begin{table}[htbp] 
    \caption{\label{tab:6}band width的选取}
    \centering
    \begin{tabular}{ccccccc} 
     \toprule 
     band width         & 1       & 3       & 5       & 6       & 7       & 10\\
    \midrule
     字准确率 & 94.19 \% & 95.03 \% & 95.19 \% & 95.22 \% & 95.22 \% & 95.23 \% \\
     句准确率 & 68.20 \% & 72.20 \% & 73.00 \% & 73.20 \% & 73.30 \% & 73.30 \%\\
     运行时间(s) & 33.5    & 92.5    & 151.1   & 185.9   & 216.9   & 301.4 \\
     \bottomrule 
    \end{tabular}
\end{table}

$5\sim 7$是band width的合理选择。

\section{总结反思}

n-gram的原理很简单,但是真正实现起来要考虑的细节实在是太多了,
这让我深刻体会到了NLP的高度复杂。
关于如何进一步改进程序,我的想法如下:

\begin{itemize}
    \item 降低n元字模型的内存消耗。在本项目中,我选用Python内置的字典(dict, 基于hash)存储状态转移矩阵,
    这实际上是一种以空间换时间的策略。
    \item 将n元字模型推广到n元词模型。词模型应比相应的字模型有更好的预测性能,但消耗的计算资源也更大。
    \item 设计GUI。
    \item 改用深度学习。拼音转汉字有几乎无限的训练语料,
    使用端到端的深度学习模型(如BERT)应能取得很好的效果,且不需要太多的特征工程。
\end{itemize}
\end{document}