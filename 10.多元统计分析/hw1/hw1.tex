%%%%%%%%%%%%%%%%%%%%%%%%%%%%%%%%%%%%%%%%%%%%%%%%%%%%%%%%%%%%%%%%%%%%%%%%%%%%%%%%%%%%
% Do not alter this block (unless you're familiar with LaTeX
\documentclass[UTF8,12pt]{article}
\usepackage[margin=1in]{geometry} 
\usepackage{amsmath,amsthm,amssymb,amsfonts, fancyhdr, color, comment, graphicx, environ}
\usepackage{xcolor}
\usepackage{mdframed}
\usepackage[shortlabels]{enumitem}
\usepackage{indentfirst}
\usepackage{hyperref}
\usepackage{xeCJK} %导入中文包
\hypersetup{
	colorlinks=true,
	linkcolor=blue,
	filecolor=magenta,      
	urlcolor=blue,
}

\pagestyle{fancy}

\newenvironment{problem}[2][Problem]
{ \begin{mdframed}[backgroundcolor=gray!20] \textbf{#1 #2}}
	{  \end{mdframed}}

% Define solution environment
\newenvironment{solution}
{\textit{Proof:}}
{}
\newenvironment{answer}
{}
{}
\newenvironment{eq}
{
	\begin{equation}
		\begin{aligned}\nonumber
}
{
		\end{aligned}
	\end{equation}
}

% prevent line break in inline mode
\binoppenalty=\maxdimen
\relpenalty=\maxdimen

%%%%%%%%%%%%%%%%%%%%%%%%%%%%%%%%%%%%%%%%%%%%%
%Fill in the appropriate information below
\lhead{张鹤潇 2018011365}
\rhead{多元统计分析} 
\chead{\textbf{Homework 1}}
%%%%%%%%%%%%%%%%%%%%%%%%%%%%%%%%%%%%%%%%%%%%%

\begin{document}
\renewcommand{\qed}{\quad\qedsymbol}
\begin{problem}{4.4}
\end{problem}
\begin{answer}
(1).$X\sim N_3(\mu, \Sigma)$, $\mu = (2, -3, 1)^T, \Sigma = 
\begin{bmatrix}
    1 & 1 & 1 \\
    1 & 3 & 2 \\
    1 & 2 & 2
    \end{bmatrix}
$.\\
记 $b=(3,-2,1)^T$, $b^TX\sim N(b^T\mu, b^T\Sigma b)$, \\
因$b^T\mu=13,b^T\Sigma b=9$, 故 $b^T X\sim N(13,3^2)$.\\
(2).令 $a=(a_1, a_2)^T$, $X_2$ 与 $Y=X_2 - a^T
\begin{bmatrix}
    X_1 \\
    X_3 
\end{bmatrix}
$ 独立当且仅当 $Cov(X, Y)=0$.\\
\begin{eq}
	Cov(X, Y)&= Var X_2 - a_1 Cov(X_2,X_1) -a_2Cov(X_2, X_3)\\
	&= 3 - a_1 -2 a_2
\end{eq}
所以可取 $a = (1,1)$.
\end{answer}

\begin{problem}{4.13}
\end{problem}
\begin{answer}
	(a).对$\Sigma$分块,
	$\Sigma=\begin{bmatrix}
		I &  0\\
		-\Sigma_{22}^{-1}\Sigma_{21}& I
	\end{bmatrix}
	$,有$
	\det\begin{bmatrix}
		I & -\Sigma_{12}\Sigma_{22}^{-1} \\
		0 & I
	\end{bmatrix}=\det\begin{bmatrix}
		I &  0\\
		-\Sigma_{22}^{-1}\Sigma_{21}& I
	\end{bmatrix}=1$
		\begin{eq}
			\begin{bmatrix}
				I & -\Sigma_{12}\Sigma_{22}^{-1} \\
			    0 & I
			\end{bmatrix}
			\begin{bmatrix}
				\Sigma_{11} & \Sigma_{12} \\
				\Sigma_{21} & \Sigma_{22}
			\end{bmatrix}
			\begin{bmatrix}
				I &  0\\
			    -\Sigma_{22}^{-1}\Sigma_{21}& I
			\end{bmatrix} &= \begin{bmatrix}
				\Sigma_{11}-\Sigma_{12}\Sigma_{22}^{-1}\Sigma_{21} &  0\\
			    0 & \Sigma_{22}
			\end{bmatrix}
		\end{eq}
		两边取行列式,得$|\Sigma|=|\Sigma_{22}|
		|\Sigma_{11}-\Sigma_{12}\Sigma_{22}^{-1}\Sigma_{21}|$.\\
	(b).由(a),
	\begin{eq}
		\Sigma^{-1} &=  \begin{bmatrix}
			I &  0\\
			-\Sigma_{22}^{-1}\Sigma_{21}& I
		\end{bmatrix}
		\begin{bmatrix}
			(\Sigma_{11}-\Sigma_{12}\Sigma_{22}^{-1}\Sigma_{21})^{-1} &  0\\
			0 & \Sigma_{22}^{-1}
		\end{bmatrix}
		\begin{bmatrix}
			I & -\Sigma_{12}\Sigma_{22}^{-1} \\
			0 & I
		\end{bmatrix}\\
		(x-\mu)^T\Sigma^{-1}(x-\mu) 
		&= \begin{bmatrix}
			x_1-\mu_1-\Sigma_{12}\Sigma_{22}^{-1}(x_2-\mu_2) \\
			x_2 - \mu_2
		\end{bmatrix}^T
		\begin{bmatrix}
			(\Sigma_{11}-\Sigma_{12}\Sigma_{22}^{-1}\Sigma_{21})^{-1} &  0\\
			0 & \Sigma_{22}^{-1}
		\end{bmatrix}\\
		&\ \begin{bmatrix}
			x_1-\mu_1-\Sigma_{12}\Sigma_{22}^{-1}(x_2-\mu_2) \\
			x_2 - \mu_2
		\end{bmatrix}\\
		&=(x_1-\mu_1-\Sigma_{12}\Sigma_{22}^{-1}(x_2-\mu_2))^T 
		(\Sigma_{11}-\Sigma_{12}\Sigma_{22}^{-1}\Sigma_{21})^{-1}\\
		&\ (x_1-\mu_1-\Sigma_{12}\Sigma_{22}^{-1}(x_2-\mu_2))
		+ (x_2 - \mu_2)^T\Sigma_{22}^{-1} (x_2 - \mu_2)
	\end{eq}
	(c).由(a)(b),$X_2\sim N(\mu_2,\Sigma_{22}),
	X_1|X_2 \sim N(\mu_1 +\Sigma_{12}\Sigma_{22}^{-1}(x_2-\mu_2), 
	\Sigma_{11}-\Sigma_{12}\Sigma_{22}^{-1}\Sigma_{21})$
\end{answer}

\begin{problem}{4.16}
\end{problem}
\begin{answer}
	\begin{eq}
		\sum_{j=1}^n (x_j-\bar{x})(\bar{x}-\mu)^T
		&= \sum_{j=1}^n (x_j-n\bar{x})](\bar{x}-\mu)^T\\
		&= (n\bar{x}-n\bar{x})(\bar{x}-\mu)^T\\
		&= 0_{p\times p}
	\end{eq}
	同理,$\sum_{j=1}^n (\bar{x}-\mu)(x_j-\bar{x})^T = 0_{p\times p}$.
\end{answer}

\begin{problem}{P25}
\end{problem}
\begin{answer}
	\begin{eq}
		L(\mu,\Sigma) &=
		\frac{1}{(2\pi)^{np/2}|\Sigma|^{n/2}}exp[-\frac{\sum_{j=1}^{n} 
		(x_j-\mu)^T \Sigma^{-1} (x_j-\mu)}{2}]\\
		l(\mu,\Sigma)&=-\frac{np}{2}\ln{2\pi} - \frac{n}{2}\ln{|\Sigma|}
		-\frac{\sum_{j=1}^{n} (x_j-\mu)^T \Sigma^{-1} (x_j-\mu)}{2}\\
		\nabla_\mu l &= \frac{1}{2} \sum_{j=1}^n \{\Sigma^{-1}(x_j-\mu)+
		[(x_j-\mu)^T \Sigma^{-1}]^T\}\\
		&= \Sigma^{-1} (\sum_{j=1}^n x_j - n\mu)\\
		\nabla_\Sigma l&= \frac{n\Sigma^{-1}}{2} - 
		\frac{1}{2}\sum_{j=1}^n x_j \nabla_\Sigma tr[(x_j-\mu)^T \Sigma^{-1} (x_j-\mu)]\\
		&= \frac{n\Sigma^{-1}}{2}- 
		\frac{1}{2}\sum_{j=1}^n x_j \nabla_\Sigma tr[ \Sigma^{-1} (x_j-\mu) (x_j-\mu)^T]\\
		&= \frac{\Sigma^{-1}}{2}[\sum_{j=1}^n (x_j-\mu)(x_j-\mu)^T\Sigma^{-1} - n]\\
		&= \frac{n\Sigma^{-1}}{2}[\frac{1}{n}\sum_{j=1}^n (x_j-\mu)(x_j-\mu)^T -\Sigma]\Sigma^{-1}
	\end{eq}
	故$\mu_{mle}=\bar{x},\Sigma_{mle}= \frac{1}{n}\sum_{j=1}^n (x_j-\bar{x})(x_j-\bar{x})^T$.
\end{answer}
\end{document}