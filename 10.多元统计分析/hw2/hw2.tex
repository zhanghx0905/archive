%%%%%%%%%%%%%%%%%%%%%%%%%%%%%%%%%%%%%%%%%%%%%%%%%%%%%%%%%%%%%%%%%%%%%%%%%%%%%%%%%%%%
% Do not alter this block (unless you're familiar with LaTeX
\documentclass[UTF8,12pt]{article}
\usepackage[margin=1in]{geometry} 
\usepackage{amsmath,amsthm,amssymb,amsfonts, fancyhdr, color, comment, graphicx, environ}
\usepackage{xcolor}
\usepackage{mdframed}
\usepackage[shortlabels]{enumitem}
\usepackage{indentfirst}
\usepackage{hyperref}
\usepackage{xeCJK} %导入中文包
\hypersetup{
	colorlinks=true,
	linkcolor=blue,
	filecolor=magenta,      
	urlcolor=blue,
}
\linespread{1.5}
\pagestyle{fancy}

\newenvironment{problem}[2][Problem]
{ \begin{mdframed}[backgroundcolor=gray!20] \textbf{#1 #2}}
	{  \end{mdframed}}

% Define solution environment
\newenvironment{solution}
{\textit{Proof:}}
{}
\newenvironment{answer}
{}
{}
\newenvironment{eq}
{
	\begin{equation}
		\begin{aligned}\nonumber
}
{
		\end{aligned}
	\end{equation}
}

% prevent line break in inline mode
\binoppenalty=\maxdimen
\relpenalty=\maxdimen

%%%%%%%%%%%%%%%%%%%%%%%%%%%%%%%%%%%%%%%%%%%%%
%Fill in the appropriate information below
\lhead{张鹤潇 2018011365}
\rhead{多元统计分析} 
\chead{\textbf{Homework 2}}
%%%%%%%%%%%%%%%%%%%%%%%%%%%%%%%%%%%%%%%%%%%%%

\begin{document}
\renewcommand{\qed}{\quad\qedsymbol}
\begin{problem}{4.18}
\end{problem}
\begin{answer}
	$$
	\mu_{mle}=\bar{X}=(4,6)^T, \Sigma_{mle}=
	\frac{1}{4}\sum_{i=1}^4 (X-\bar{X})(X-\bar{X})^T=
	\begin{bmatrix}
		0.5 & 0.25\\
		0.25 & 1.5
	\end{bmatrix}
	$$
\end{answer}

\begin{problem}{4.19}
\end{problem}
\begin{answer}
	(a).$(X_1-\mu)^T\Sigma^{-1}(X_1-\mu)\sim \chi^2_6$.\\
	(b).$\bar{X}\sim N_6(\mu, \frac{1}{20}\Sigma), 
	\sqrt{n}(\bar{X}-\mu)\sim N_6(0,\Sigma)$.\\
	(c).$(n-1)S \sim W_6(19,\Sigma)$
\end{answer}
\begin{problem}{4.20(b)}
\end{problem}
\begin{answer}
	$$
	B =\begin{bmatrix}
		1 & 0 & 0 & 0 & 0 & 0\\
		0 & 0 & 1 & 0 & 0 & 0
	\end{bmatrix}
	,B(19S)B^T=19\begin{bmatrix}
		s_{11} & s_{13}\\
		s_{31} & s_{33}
	\end{bmatrix}
	$$
	记$\Sigma_{13}=\begin{bmatrix}
		\sigma_{11} & \sigma_{13}\\
		\sigma_{31} & \sigma_{33}
	\end{bmatrix}
	$,则$B(19S)B^T \sim W_2(19,\Sigma_{13})$.
\end{answer}

\begin{problem}{5.1}
\end{problem}
\begin{answer}
	(a).
	$$
	\bar{X}=(6,10)^T, s=\begin{bmatrix}
		8 & -10/3\\
		-10/3 & 2
	\end{bmatrix}
	$$
	$$
	T^2 = 4(\bar{X}-\mu)^T s^{-1}(\bar{X}-\mu)=13.6
	$$
	(b).$T^2\sim \frac{2(4-1)}{4-2}F_{2,4-2}=3F_{2,2}$\\
	(c).$3F_{2,2}(0.05)=57>T^2$, 无法拒绝$H_0$.
\end{answer}

\begin{problem}{5.2}
\end{problem}
\begin{answer}
$$
Y=XC^T = \begin{bmatrix}
	-10 & 14\\
	-1 & 17\\
	-3 & 15\\
	-2 & 18
\end{bmatrix},\mu_Y=C\mu=(-4,16)^T
$$
计算可得$T^2=4(\bar{Y}-\mu_Y)S_Y^{-1}(\bar{Y}-\mu_Y)^T=13.6$,
保持不变。
\end{answer}

\begin{problem}{5.5}
\end{problem}
	$T^2=42(\bar{X}-\mu)S^{-1}(\bar{X}-\mu)^T=1.17 
	$\\
	而$\frac{n(n-1)}{n-p}=2.05,
	2.05F_{p,n-p}(\alpha)=2.05F_{2,40}(0.05)=6.62>T^2$.\\
	无法拒绝$H_0$.\ 而由图5.1,可知$\mu$处于椭圆内部,也即$\mu$在置信域内。
\end{document}