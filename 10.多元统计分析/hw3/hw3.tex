%%%%%%%%%%%%%%%%%%%%%%%%%%%%%%%%%%%%%%%%%%%%%%%%%%%%%%%%%%%%%%%%%%%%%%%%%%%%%%%%%%%%
% Do not alter this block (unless you're familiar with LaTeX
\documentclass[UTF8,12pt]{article}
\usepackage[margin=1in]{geometry} 
\usepackage{amsmath,amsthm,amssymb,amsfonts, fancyhdr, color, comment, graphicx, environ}
\usepackage{xcolor}
\usepackage{mdframed}
\usepackage[shortlabels]{enumitem}
\usepackage{indentfirst}
\usepackage{hyperref}
\usepackage{float}
\usepackage{xeCJK} %导入中文包
\hypersetup{
	colorlinks=true,
	linkcolor=blue,
	filecolor=magenta,      
	urlcolor=blue,
}
\linespread{1.5}
\pagestyle{fancy}

\newenvironment{problem}[2][Problem]
{ \begin{mdframed}[backgroundcolor=gray!20] \textbf{#1 #2}}
	{  \end{mdframed}}

% Define solution environment
\newenvironment{Proof}
{\textit{Proof:}}
{}
\newenvironment{answer}
{%\textit{Answer:}
}
{}
\newenvironment{eq}
{
	\begin{equation}
		\begin{aligned}\nonumber
}
{
		\end{aligned}
	\end{equation}
}

% prevent line break in inline mode
\binoppenalty=\maxdimen
\relpenalty=\maxdimen

%%%%%%%%%%%%%%%%%%%%%%%%%%%%%%%%%%%%%%%%%%%%%
%Fill in the appropriate information below
\lhead{张鹤潇 2018011365}
\rhead{多元统计分析} 
\chead{\textbf{Homework 3}}
%%%%%%%%%%%%%%%%%%%%%%%%%%%%%%%%%%%%%%%%%%%%%

\begin{document}
\renewcommand{\qed}{\quad\qedsymbol}

这里只列出结果,代码详见\href{.\\hm3.rmd}{hw3.rmd}。

\begin{problem}{5.7}
\end{problem}
\begin{answer}
	$\mu_1,\mu_2,\mu_3$的Bonferroni置信区间$[3.64,5.64],
	[37.10,53.70],[8.85,11.08]$,\\
	联合$T^2$置信区间$[3.40,5.88],[35.05,55.75],[8.57,11.36]$. $T^2$置信区间更大。
	\begin{figure}[H]
		\centering
		\includegraphics[scale=0.6]{QQPlot.png}
		\caption{Q-Q plot}
	\end{figure}

	\begin{figure}[H]
		\centering
		\includegraphics[scale=0.6]{scatterplot.png}
		\caption{scatter plot matrix}
	\end{figure}

	\begin{figure}[H]
		\centering
		\includegraphics[scale=0.5]{chisplot.png}
		\caption{Chi-square plot}
	\end{figure}

	由以上各图可知,$x1,x2,x3$的分布都是有偏的。正态性假设不合理。
\end{answer}

\begin{problem}{6 Q1}
\end{problem}
\begin{answer}
	设想如下两种情形。\\
	(1).比较两种药物的效果差异。对于一群病人(上百个)中的每一位,
	比较其服药后一段时间内血液中血细胞含量的变化。这种在同一个病人身上的成对比较是有意义的。\\
	(2).比较两家工厂生产的元件质量差异。在两家工厂生产的同类元件中,各抽取一百个,统计每个元件的重量,
	这时成对比较就不合理,因为一对元件之间没有事实上的对应关系。
\end{answer}

\begin{problem}{6 Q2}
\end{problem}
\begin{answer}
	(a).$H_0:\mu_1-\mu_2=0,\alpha=0.05$, 根据已有数据,
	\begin{eq}
		S_{pooled}&=\frac{(n_1-1)S_1+(n_2-1)S_2}{n_1+n_2-2}
		=\begin{bmatrix}
			10963.69 & 21505.42\\
			21505.42 & 63661.31
		\end{bmatrix}\\
		T^2&=(\frac{1}{n_1}+\frac{1}{n_2})^{-1} (\bar{X_1}-\bar{X_2})^T
		S_{pooled}^{-1} (\bar{X_1}-\bar{X_2})\\
		&= 16.1\\
		c &= \frac{p(n_1+n_2-2)}{n_1+n_2-p-1}F_{p,n_1+n_2-p-1}(\alpha) = 6.2 <T^2
	\end{eq}
	拒绝$H_0$. 对拒绝$H_0$起关键作用的均值线性组合:
	\begin{eq}
		S_{pooled}^{-1}(\bar{X_1}-\bar{X_2})=(0.0017,0.0026)^T
	\end{eq}
	(b).
	\begin{eq}
		T^2 &= (\bar{X_1}-\bar{X_2})^T (S_1/n_1+S_2/n_2)^{-1} (\bar{X_1}-\bar{X_2})
		&= 15.7\\
		c &= \chi^2_p(\alpha) = 6.0 < T^2
	\end{eq}
	拒绝$H_0$. 对拒绝$H_0$起关键作用的均值线性组合:
	\begin{eq}
		(S_1/n_1+S_2/n_2)^{-1} (\bar{X_1}-\bar{X_2})=(0.041,0.063)^T
	\end{eq}
	(c).根据Ch6.6,对$H_0:\Sigma_1=\Sigma_2$检验如下:
	\begin{eq}
		v &=\frac{1}{2}p(p+1)(g-1)=3\\
		u &= \frac{2p^2+3p-1}{6(p+1)(g-1)}
		[\sum_l \frac{1}{n_l-1}-\frac{1}{\sum_l (n_l-1)}]
		=0.022\\
		C &= (1-u)\{[\sum_l (n_l-1)\ln{|S_p|}-\sum_l[(n_l-1)]\ln{|S_l|}]\}
		= 18.9\\
		\chi^2_v(\alpha) &= 6.0 < C
	\end{eq}
	拒绝$H_0$.\\
	(d).既然协方差矩阵相等的假设不成立,(b)的结果更可信。
\end{answer}

\begin{problem}{6.11}
\end{problem}
\begin{answer}
	\begin{eq}
		L(\mu_1,\mu_2,\Sigma)&=\frac{1}{(2\pi)^{\frac{p(n_1+n_2)}{2}} 
		|\Sigma|^{\frac{n_1+n_2}{2}}}\\
		&\ \exp{\{-\frac{1}{2} [\sum_{i=1}^{n_1}
		(x_i-\mu_1)^T \Sigma^{-1}(x_i-\mu_1) + \sum_{j=1}^{n_2} 
		(x_j-\mu_2)^T \Sigma^{-1}(x_j-\mu_2)]\}}\\
		l &=\log{L} = -\frac{1}{2} [\sum_{i=1}^{n_1}
		(x_i-\mu_1)^T \Sigma^{-1}(x_i-\mu_1) + \sum_{j=1}^{n_2} 
		(x_j-\mu_2)^T \Sigma^{-1}(x_j-\mu_2)]\\
		&\ - \frac{p(n_1+n_2)}{2} \log{2\pi} - \frac{n_1+n_2}{2}\log{|\Sigma|}\\
	\end{eq}
	对$\mu_1,\mu_2,\Sigma$分别求导,其中$\mu_1,\mu_2$的计算过程与单总体情形一致,此处略去过程。
	结果为$\hat{\mu_1}=\bar{X_1},\hat{\mu_2}=\bar{X_2}$.
	\begin{eq}
		\nabla_\Sigma l &=-\frac{n_1+n_2}{2}\Sigma^{-1}+\frac{1}{2}[
		\sum_{i=1}^{n_1} \Sigma^{-1} (x_i-\mu_1)(x_i-\mu_1)^T\Sigma^{-1} +
		\sum_{j=1}^{n_2} \Sigma^{-1} (x_j-\mu_2)(x_j-\mu_2)^T\Sigma^{-1}]\\
		&= \frac{\Sigma^{-1}}{2} \{-(n_1+n_2) + 
		[\sum_{i=1}^{n_1}(x_i-\mu_1)(x_i-\mu_1)^T +
		\sum_{j=1}^{n_2}(x_i-\mu_1)(x_i-\mu_1)^T]\Sigma^{-1}\}\\
		&= \frac{\Sigma^{-1}}{2} \{-(n_1+n_2)] + [(n_1-1)S_1+(n_2-2)S_2]\Sigma^{-1}\}
	\end{eq}
	令$\nabla_\Sigma l = 0$, 得$\hat{\Sigma}=\frac{(n_1-1)S_1+(n_2-2)S_2}{n_1+n_2}$. 
\end{answer}
\end{document}