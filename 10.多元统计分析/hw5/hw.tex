%%%%%%%%%%%%%%%%%%%%%%%%%%%%%%%%%%%%%%%%%%%%%%%%%%%%%%%%%%%%%%%%%%%%%%%%%%%%%%%%%%%%
% Do not alter this block (unless you're familiar with LaTeX
\documentclass[UTF8,12pt]{article}
\usepackage[margin=1in]{geometry} 
\usepackage{amsmath,amsthm,amssymb,amsfonts, fancyhdr, color, comment, graphicx, environ}
\usepackage{xcolor}
\usepackage{mdframed}
\usepackage[shortlabels]{enumitem}
\usepackage{indentfirst}
\usepackage{hyperref}
\usepackage{float}
\usepackage{xeCJK} %导入中文包
\hypersetup{
	colorlinks=true,
	linkcolor=blue,
	filecolor=magenta,      
	urlcolor=blue,
}
\linespread{1.5}
\pagestyle{fancy}

\newenvironment{problem}[2][Problem]
{ \begin{mdframed}[backgroundcolor=gray!20] \textbf{#1 #2}}
	{  \end{mdframed}}

% Define solution environment
\newenvironment{Proof}
{\textit{Proof:}}
{}
\newenvironment{answer}
{%\textit{Answer:}
}
{}
\newenvironment{eq}
{
	\begin{equation}
		\begin{aligned}\nonumber
}
{
		\end{aligned}
	\end{equation}
}

% prevent line break in inline mode
\binoppenalty=\maxdimen
\relpenalty=\maxdimen

%%%%%%%%%%%%%%%%%%%%%%%%%%%%%%%%%%%%%%%%%%%%%
%Fill in the appropriate information below
\lhead{张鹤潇 2018011365}
\rhead{多元统计分析} 
\chead{\textbf{Homework 5}}
%%%%%%%%%%%%%%%%%%%%%%%%%%%%%%%%%%%%%%%%%%%%%

\begin{document}
\renewcommand{\qed}{\quad\qedsymbol}
% \setlength{\parindent}{0pt}
\begin{problem}{9.19}
\end{problem}
\begin{answer}
	(1). $m=3$时,PC方法:
	\begin{eq}
		L_z = \begin{bmatrix}
			0.97 & -0.11 & \\
			0.94 &  & -0.31\\
			0.95 &  & 0.14\\
			0.66 & 0.65 & 0.32\\
			0.78 & 0.29 & \\
			0.65 & -0.62 & 0.43\\
			0.91 & -0.19 & -0.31
		\end{bmatrix},
		\phi_z = diag(0.039, 0.013, 0.087, 0.045, 0.305, 0.012, 0.033)
	\end{eq}MLE方法:
	\begin{eq}
		L_z = \begin{bmatrix}
			0.90 & 0.38 &\\
			0.78 & 0.60 &\\
			0.93 & 0.20 &\\
			0.73 & -0.12 & 0.67\\
			0.69 & 0.23 & 0.17\\
			0.76 & -0.13 & -0.64\\
			0.76 & 0.61 & 0.11
		\end{bmatrix},
		\phi_z = diag(0.039, 0.034, 0.088, 0.005, 0.447, 0.005, 0.038)
	\end{eq}
	(2). $m=2$时,用PC方法得到的$L_z$:
	$$
	\begin{bmatrix}
		0.97 & -0.11 \\
			0.94 &  \\
			0.95 &  \\
			0.66 & 0.65 \\
			0.78 & 0.29 \\
			0.65 & -0.62 \\
			0.91 & -0.19 
	\end{bmatrix}
	$$
	(3). $L=V^{1/2}L_z,\phi=V^{1/2}\phi V^{1/2}$,结果见代码文件。

	(4). 在(1)基础上,用varimax因子旋转法。
	$$
	L_z = \begin{bmatrix}
		0.79 & 0.37 & 0.44\\
		0.91 & 0.32 & 0.19\\
		0.65 & 0.54 & 0.44\\
		0.26 & 0.96 & \\
		0.54 & 0.47 & 0.21\\
		0.30 &  & 0.95\\
		0.92 & 0.18 & 0.30
	\end{bmatrix},
	\phi_z = diag(0.039, 0.034, 0.088, 0.005, 0.447, 0.005, 0.038)
	$$
	可见$L_z$发生了变化,$\phi_z$不变。\\
	a). The specific variances = $(0.039, 0.034, 0.088, 0.005, 0.447, 0.005, 0.038)$;\\
	b). The communalities = $(0.961,0.966,0.912,0.995,0.553,0.995,0.962)$;\\
	c). Variance explained by 1st factor = $3.175$;\\
	d). $p=2.01\times 10^{-13}$.\\
	(5). 用regression方法,因子得分$(-0.79,-0.36,-0.49)$.
\end{answer}

\end{document}