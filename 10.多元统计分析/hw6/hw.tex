%%%%%%%%%%%%%%%%%%%%%%%%%%%%%%%%%%%%%%%%%%%%%%%%%%%%%%%%%%%%%%%%%%%%%%%%%%%%%%%%%%%%
% Do not alter this block (unless you're familiar with LaTeX
\documentclass[UTF8,12pt]{article}
\usepackage[margin=1in]{geometry} 
\usepackage{amsmath,amsthm,amssymb,amsfonts, fancyhdr, color, comment, graphicx, environ}
\usepackage{xcolor}
\usepackage{mdframed}
\usepackage[shortlabels]{enumitem}
\usepackage{indentfirst}
\usepackage{hyperref}
\usepackage{float}
\usepackage{xeCJK} %导入中文包
\usepackage[linesnumbered, boxed, ruled, commentsnumbered, noend]{algorithm2e}

\hypersetup{
	colorlinks=true,
	linkcolor=blue,
	filecolor=magenta,      
	urlcolor=blue,
}
\linespread{1.5}
\pagestyle{fancy}

\newenvironment{problem}[2][Problem]
{ \begin{mdframed}[backgroundcolor=gray!20] \textbf{#1 #2}}
	{  \end{mdframed}}


% Define solution environment
\newenvironment{Proof}
{\textit{Proof:}}
{}
\newenvironment{answer}
{%\textit{Answer:}
}
{}
\newenvironment{eq}
{
	\begin{equation}
		\begin{aligned}\nonumber
}
{
		\end{aligned}
	\end{equation}
}
\usepackage{listings}
\usepackage{xcolor}


% prevent line break in inline mode
\binoppenalty=\maxdimen
\relpenalty=\maxdimen

%%%%%%%%%%%%%%%%%%%%%%%%%%%%%%%%%%%%%%%%%%%%%
%Fill in the appropriate information below
\lhead{张鹤潇 2018011365}
\rhead{多元统计分析} 
\chead{\textbf{Homework 6}}
%%%%%%%%%%%%%%%%%%%%%%%%%%%%%%%%%%%%%%%%%%%%%

\begin{document}

\renewcommand{\qed}{\quad\qedsymbol}
%\setlength{\parindent}{0pt}

\begin{problem}{11.4}
\end{problem}
\begin{answer}
    (1).
\begin{eq}
    R_1: \frac{f_1(x)}{f_2(x)}\ge 
    \frac{c(1|2)p_2}{c(2|1)p_1} = 0.4\\
    R_2: \frac{f_1(x)}{f_2(x)}<
    \frac{c(1|2)p_2}{c(2|1)p_1} = 0.4
\end{eq}(2).
\begin{eq}
        \frac{f_1(x)}{f_2(x)} = 0.6 \ge 0.4
    \end{eq}故此新项属于$\pi_1$.
\end{answer}

\begin{problem}{11.14}
\end{problem}
\begin{answer}
在例11.3中,$\hat{a}=(37.6, -28.9)^T, \hat{m} = \hat{a}\frac{1}{2}
(\bar{y_1}+\bar{y_2})=-4.61$.\\
根据(11-21),
\begin{eq}
    \hat{a}^* &= \frac{\hat{a}}{||\hat{a}||} = (0.79,-0.61)^T\\
    \hat{m}^* &= \frac{\hat{m}}{||\hat{a}||} = -0.10
\end{eq}因$x_0^T \hat{a}^* = -0.14 < \hat{m}^*$,将$x_0$分到$\pi_2$.

根据(11-22),
\begin{eq}
    \hat{a}^* &= \frac{\hat{a}}{\hat{a}_1} = (1,-0.77)^T\\
    \hat{m}^* &= \frac{\hat{m}}{\hat{a}_1} = -0.12
\end{eq}因$ x_0^T \hat{a}^* = -0.18 < \hat{m}^*$,将$x_0$分到$\pi_2$.

结果与例11.3中刻度变换之前的一致。应该是一致的,因为不等式两边均乘以相同的变换系数。
\end{answer}
\end{document}